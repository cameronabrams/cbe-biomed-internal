\documentclass[10pt]{beamer}
\input{header}

\title{An introduction to modern molecular simulation methods in the biosciences}
\date{January 16, 2024}
\author{Cameron F. Abrams, \href{mailto:cfa22@drexel.edu}{cfa22@drexel.edu}\\
Bartlett '81 -- Barry '81 Professor and Head }
\institute{Department of Chemical and Biological Engineering}
\titlegraphic{\hfill\includegraphics[height=0.75cm]{drexel-horz-blue.png}}

\begin{document}

\maketitle

\section{Introduction}

\input{frame_feynman_quote}

\input{frame_md_movies}

\section{Molecular Dynamics}

\input{frame_md}

\input{frame_motiv1}

\input{frame_essential_problem}

\section{Computational Drug Design}

\section{Summary}

\begin{frame}[fragile]{Summary}
\begin{itemize}
\item \textcolor{red}{OTFP} combines enhanced sampling and free-energy-profile generation to provide deeper understanding of biomolecular mechanisms
\begin{itemize}
\item We rationalized DAVEI activity dependence on linker length
\item We recapitulate water-mediated ``knock-on'' transport of multiple K$^+$ ions in Kv1.2, but also predict reduced barriers for dry transport
\end{itemize}
\item Demonstrated the \textcolor{magenta!80!black}{Climbing Multistrings Method} for locating stationary points in feature space
\item Demonstrated \textcolor{green!80!black}{Markovian milestoning} along minimum free-energy pathways to estimate entry and exit rates of O$_{\sf 2}$ in MSOX
\begin{itemize}
\item MSOX: Simulations agree with experiment that modified ping-pong mechanism is more likely than ping-pong because substrate influences O$_{\sf 2}$
entry and exit kinetics
%\item MSOX: Substrate-induced gatekeeper closing and active-site desolvation likely underly the substrates effect on O$_{\sf 2}$.
\end{itemize}
\end{itemize}
\end{frame}

% \input{frame_ack}

\end{document}
