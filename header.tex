\usepackage{amsmath}
\usepackage{amssymb}
%\usepackage{dsfont}
\usepackage{accents}
\usepackage{hyphenat}
\usepackage{multirow}
\usepackage{hyperref}
\usetheme[progressbar=frametitle]{metropolis}
\usepackage{appendixnumberbeamer}
\usefonttheme[onlymath]{serif}
\usepackage{booktabs}
\usepackage[scale=2]{ccicons}
\usepackage{animate}
\usepackage{xcolor}
\usepackage{soul}

%\usepackage{pgfplots}
%\usepgfplotslibrary{dateplot}
%\usepgflibrary{arrows}

\usepackage{xspace}
\newcommand{\themename}{\textbf{\textsc{metropolis}}\xspace}
\usetikzlibrary{snakes,arrows,mindmap,trees,backgrounds,shapes.geometric}
\tikzset{graphics/.style={inner sep=0,outer sep=0}}
\tikzset{scaleall/.style={scale=#1, transform shape}}



\newlength{\bbw}
\newlength{\bbh}
\newcommand{\pcuad}[3][]{
  % First arguments is an optional prefix for the coordinate names

  \setlength{\bbw}{#2}
  \setlength{\bbh}{#3}
  \useasboundingbox(0,0) rectangle (\bbw,\bbh);

  \path(.5\bbw,.5\bbh) coordinate(#1cp);
  \path(\bbw,0)        coordinate(#1se);
  \path(0,0)           coordinate(#1sw);
  \path(0,\bbh)        coordinate(#1nw);
  \path(\bbw,\bbh)     coordinate(#1ne);
 
  
  \path(\bbw,.5\bbh) coordinate(#1ep);
  \path(.5\bbw,0)    coordinate(#1sp);
  \path(.5\bbw,\bbh) coordinate(#1np);
  \path(0,.5\bbh)    coordinate(#1wp);
       
  \path(.75\bbw,.5\bbh) coordinate(#1hr);
  \path(.25\bbw,.5\bbh) coordinate(#1hl);
  \path(.5\bbw,.25\bbh) coordinate(#1vl);
  \path(.5\bbw,.75\bbh) coordinate(#1vu);

  \path(.75\bbw,.75\bbh) coordinate(#1c1);
  \path(.25\bbw,.75\bbh) coordinate(#1c2);
  \path(.25\bbw,.25\bbh) coordinate(#1c3);
  \path(.75\bbw,.25\bbh) coordinate(#1c4);


%  +-----------------------------------------------+
%  |nw                    np                     ne|
%  |                                               |
%  |                                               |
%  |          c2          vu          c1           |
%  |                                               |
%  |                                               |
%  |wp        hl          cp          hr         ep|
%  |                                               |
%  |                                               |
%  |          c3          vl          c4           |
%  |                                               |
%  |                                               |
%  |sw                    sp                     se|
%  +-----------------------------------------------+
}
 
        
\newcommand{\showcuad}[1][]{

   \draw[help lines,xstep=.5,ystep=.5,gray!10] (#1sw) grid (#1ne);
   \draw[help lines,xstep=1,ystep=1,gray]      (#1sw) grid (#1ne);
   %\draw[help lines,xstep=.25,ystep=.25,gray!20] (sw) grid (ne);
   % \draw[help lines,xstep=1,ystep=1,gray] (sw) grid (ne);
   % \foreach \x in {-20,-14.5,...,20} {
   %     \node [anchor=north, gray,yshift=30] at (\x,0) {\tiny \bf \x};
   %     \node [anchor=east,gray,xshift=30] at (0,\x) {\tiny \bf  \x};
   % }
               
    \fill(#1se) circle(.1) node[anchor=south east]{#1se};
    \fill(#1sw) circle(.1) node[anchor=south west]{#1sw};
    \fill(#1ne) circle(.1) node[anchor=north east]{#1ne};
    \fill(#1nw) circle(.1) node[anchor=north west]{#1nw};
                  
    \fill(#1hr) circle(.1) node[above]{#1hr};
    \fill(#1hl) circle(.1) node[above]{#1hl};
    \fill(#1vu) circle(.1) node[above]{#1vu};
    \fill(#1vl) circle(.1) node[above]{#1vl};
                  
    \fill(#1sp) circle(.1) node[anchor=south]{#1sp};
    \fill(#1wp) circle(.1) node[anchor=west] {#1wp};
    \fill(#1np) circle(.1) node[anchor=north]{#1np};
    \fill(#1ep) circle(.1) node[anchor=east] {#1ep};

    \fill(#1cp) circle(.1) node[above]{#1cp};
    \fill(#1c1) circle(.1) node[above]{#1c1};
    \fill(#1c2) circle(.1) node[above]{#1c2};
    \fill(#1c3) circle(.1) node[above]{#1c3}; 
    \fill(#1c4) circle(.1) node[above]{#1c4};

    %\fill[red](cp|-c1) circle(.1) node[anchor=north]{cp|-c1};

}

\newcommand{\ds}{\displaystyle}
\newcommand{\dsum}{\displaystyle \sum}
\newcommand{\uu}[1]{{\boldsymbol #1}}
\newcommand{\ud}{\,\mathrm{d}}
\def\ttau{\uu{\tau}}
\def\bb{\uu{b}}
\def\nb{\uu{n}}
\def\pb{\uu{p}}
\def\wb{\uu{w}}
\def\xb{\uu{x}}
\def\ab{\uu{a}}
\def\yb{\uu{y}}
\def\vb{\uu{v}}
\def\fb{\uu{f}}
\def\zb{\uu{z}}
\def\Xb{\uu{X}}
\def\etab{\uu{\eta}}
\def\thetab{\uu{\theta}}
\def\lambdab{\uu{\lambda}}
\def\gammab{\uu{\gamma}}
\def\taub{\uu{\tau}}
\def\varphib{\uu{\varphi}}
\def\Ab{\uu{A}}
\def\Bb{\uu{B}}
\def\Gb{\uu{G}}
\def\Fb{\uu{F}}
\def\Jb{\uu{J}}
\def\Rb{\uu{R}}
\def\Tb{\uu{T}}
\def\rb{\uu{r}}
\def\ab{\uu{a}}
\newcommand{\molar}[1]{\underaccent{\bar}{#1}}
\def\um{\molar{U}}
\def\hm{\molar{H}}
\def\sm{\molar{S}}
\def\am{\molar{A}}
\def\gm{\molar{G}}
\def\hh{\hat{H}}
\def\sh{\hat{S}}
\def\vm{\molar{V}}
\def\cp{C_{\rm P}}
\def\cv{C_{\rm V}}
\def\cps{C_{\rm P}^*}
\def\cvs{C_{\rm V}^*}
\def\wsd{\dot{W}_s}
\def\md{\dot{M}}\newcommand{\pd}[2]{\left(\frac{\partial {#1}}{\partial 
{#2}}\right)}
\newcommand{\tpd}[3]{\left(\frac{\partial {#1}}{\partial {#2}}\right)_{#3}}
\newcommand{\tpdn}[4]{\left(\frac{\partial^{#4} {#1}}{\partial 
{#2}^{#4}}\right)_{#3}}

